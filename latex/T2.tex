
\documentclass[12pt,a4paper,openany,oneside]{abntex2}


\usepackage[utf8]{inputenc}
\usepackage[brazil]{babel}
\usepackage{graphicx}
\usepackage{indentfirst}
\usepackage{etoolbox}
\usepackage{amsmath}
\usepackage{graphicx}
%\usepackage[pdftex]{hyperref}

\title{Implementação dos métodos numéricos Gauss-Seidel e SOR baseado em Gauss-Seidel}
\author{Carlos Eduardo Amorim Gusmão}
\date{2025}

\instituicao{%
Universidade Federal de Minas Gerais\\
Bacharelado em Engenharia Mecânica}

\tipotrabalho{Trabalho apresentado à disciplina de Métodos Numéricos aplicados a Engenharia Mecânica.}

\orientador{Prof. Lucas Ribeiro Alves}

\local{Belo Horizonte}
\data{28 de maio de 2025}



\begin{document}

\imprimircapa
\imprimirfolhaderosto

% Início do conteúdo
\begin{center}
    \textbf{\large Introdução}
\end{center}


O objetivo deste trabalho foi implementar e analisar o 
desempenho de dois métodos iterativos clássicos para a 
resolução de sistemas lineares: o método de Gauss-Seidel 
e o método de Relaxação Sucessiva (SOR - Successive Over-Relaxation). 
Esses métodos foram aplicados a sistemas lineares cujas soluções 
são conhecidas previamente, de modo a permitir a validação dos 
resultados obtidos numericamente.
A implementação foi realizada na linguagem C++, com comentários 
explicativos ao longo do código para descrever cada etapa do 
processo computacional. Utilizou-se um mesmo critério de parada 
para ambos os métodos, baseado na precisão do erro relativo entre 
iterações consecutivas, assegurando uma comparação justa entre os 
algoritmos.

Durante a execução, foram armazenados os valores dos erros a 
cada iteração, o que possibilitou a análise da taxa de 
convergência de cada método. Os resultados foram organizados 
em gráficos que representam a evolução do erro em função do 
número de iterações, permitindo uma avaliação visual da eficiência 
de convergência. A comparação entre Gauss-Seidel e SOR evidencia 
as vantagens e limitações de cada abordagem, especialmente no que 
se refere à escolha do parâmetro de relaxação no método SOR.
\\
\begin{center}
    \textbf{\large Desenvolvimento}
\end{center}

O código foi desenvolvido na linguagem C++ usando a programação modularizada, com uma função para cada etapa:

$\bullet$ Função para Gauss-Seidel

$\bullet$ Função para SOR-Gauss-Seidel

$\bullet$ Função principal main

O método de Gauss-Seidel tem como ideia principal isolar uma das variáveis
e resolver normalmente a equação gerada pelo isolamento, usando a fórmula geral:

\begin{equation*}
x_i^{(k+1)} = \frac{1}{a_{ii}} \left( b_i - \sum_{j=1}^{i-1} a_{ij} x_j^{(k+1)} - \sum_{j=i+1}^{n} a_{ij} x_j^{(k)} \right)
\nonumber
\end{equation*}

Para que esse método convirja, é necessário que o sistema seja
uma matriz diagonal dominante, ou seja, os valores em que i=j, devem ser maiores que o módulo da soma dos outros valores da mesma linha. Para tanto, foi escolhido o seguinte sistema:

\begin{equation}
\begin{cases}
4x + -y = 15 \\
-x + 4y -z = 10 \\
-y+3z = 10
\end{cases}
\nonumber
\end{equation}

Em formato de matriz, é:
\[
\left[
\begin{array}{ccc|c}
4 & -1 & 0 & 15 \\
-1 & 4 & -1 & 10 \\
0 & -1 & 3 & 10
\end{array}
\right]
\]

Diferentemente do método de Jacobi, Gauss-Seidel usa os valores imediatos encontrados
,em vez de usar valores pré definidos em toda iteração. Mas também possui chutes inciais. Aqui foram
\begin{equation} 
x=0; y=0; z=0 
\nonumber
\end{equation}

SOR baseado em Gauss-Seidel consiste em aprimorar Gauss-Seidel, usando um $\lambda$, cujo valor pode, ou não,
contribuir para que convirja mais rápido, assim chegando a solução real de forma muito mais rápido se bem ajustado.
Aqui foram feitos diversos testes para que fosse escolhido um valor aceitável para $\lambda$, chegando ao valor de 1,05.
Esse método também tem uma fórmula geral e consiste em:

\begin{equation}
    x_{\!i}^{\text{novo}} = \lambda x_{\!i}^{\text{novo}} + (1 - \lambda) x_{\!i}^{\text{velho}}
\nonumber
\end{equation}

No código o cálculo da variação da solução entre iterações é realizado por meio da variável dx, que representa a norma 
da diferença entre os vetores solução de duas iterações consecutivas. Essa variação é usada como critério de parada 
do método: quando dx se torna menor que um valor previamente definido:
\begin{equation} \varepsilon=10^-6
    \nonumber
\end{equation}

Além disso, a cada iteração, é calculado o erro da solução atual em relação ao sistema original. Esse erro é obtido por:
\begin{equation}
    ||A\vec{x}-\vec{b}||
    \nonumber
\end{equation}
Feito isso, a cada iteração realizada, são impressas as soluções, o dx e o erro.
\\
\begin{center}
    \textbf{\large Análise dos dados gerados}
\end{center}

A partir dos erros, foi gerada a tabela, usada posteriormente para gerar o gráfico.
\begin{center}
    \begin{tabular}{|c|c|c|}
\hline
Iteração & Erro (Gauss-Seidel) & Erro-SOR \\ \hline
1   & 5,64618   & 5,05988   \\ \hline
2   & 1,40683   & 1,12791   \\ \hline
3   & 0,205163  & 0,0120131 \\ \hline
4   & 0,0299195 & 0,000644079 \\ \hline
5   & 0,00436327 & 0,000130519 \\ \hline
6   & 0,00063631 & 2,73e-06 \\ \hline
7   & 9,28e-05   & 1,58e-07 \\ \hline
8   & 1,35e-05   &         \\ \hline
9   & 1,97e-06 & \\ \hline
10  & 2,88e-07 & \\ \hline
\end{tabular}
\end{center}

Foi obtido o gráfico:

\begin{figure}[!h]  % o !h tenta colocar a figura exatamente aqui
    \centering       % centraliza a imagem
    \includegraphics[width=13cm]{grafico} % insere a imagem redimensionada
    \caption{Gráfico da evolução do erro}
    \label{fig:minha_imagem}
\end{figure}

O gráfico apresenta a evolução do erro para os métodos Gauss-Seidel (representado pela linha vermelha com marcadores de losango) e SOR baseado em Gauss-Seidel (representado pela linha azul com marcadores de quadrado). O eixo horizontal mostra o "Número de iterações" e o eixo vertical representa o "Erro".
Ambos os métodos começam com um erro inicial elevado. Isso é normal, pois as iterações iniciais são geralmente as que produzem as maiores reduções de erro. Nas primeiras 2 a 3 iterações, ambos os métodos mostram uma queda abrupta e significativa no erro. O erro diminui rapidamente após a primeira iteração e a partir da quarta iteração
o método de SOR já possui um erro muito próximo de zero, enquando o Gauss-Seidel, está um pouco mais acima. Ao final, percebe-se que para o método de SOR foram necessárias menos iterações em comparação com o outro método.

\begin{center}
    \textbf{\large Conclusão}
\end{center}

O gráfico evidencia claramente a superioridade do método SOR sobre o método de Gauss-Seidel tradicional em termos de velocidade de convergência para o problema. Embora ambos os métodos sejam capazes de reduzir o erro e convergir para uma solução de alta precisão, o SOR atinge esse estado de convergência em um 
número muito menor de iterações. Essa diferença no desempenho entre os métodos ressalta a eficácia do fator de relaxamento ($\lambda$) ao método SOR na otimização do processo iterativo de Gauss-Seidel. Ao "super-relaxar" ou "sub-relaxar" as estimativas das variáveis a cada passo, o SOR consegue corrigir o 
erro de forma mais eficiente, evitando oscilações ou uma convergência muito lenta.

Portanto, a conclusão é que, para problemas com características semelhantes ao exemplificado, 
o método SOR representa uma escolha mais eficiente computacionalmente, pois permite alcançar 
a solução desejada com menor esforço, resultando em menor tempo de processamento e recursos 
computacionais. Essa é uma informação importante para a seleção de algoritmos em 
aplicações de engenharia e ciência que envolvem a solução numérica de grandes sistemas lineares.

\newpage
\begin{center}
    \textbf{\large Referências}
\end{center}

\setlength{\parindent}{0pt}
CHAPRA, STEVEN C. \textbf{Métodos numéricos para engenharia}. Tradução: Helena Castro
São Paulo, 2008. p.250-254, 27 mai. 2025.

\end{document}
